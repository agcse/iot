\documentclass{article}
\usepackage{fancyhdr}
\usepackage{datetime}
\usepackage{parskip}

\pagestyle{fancy}

\fancyhf{}
\fancyfoot[L]{University of Oulu. \today}

\lhead{Waste management system}
\rhead{Page \thepage}

\title{Exercise 1: Design for urban IoT. Waste management system
\author{Andrei Golubev, Hassan Shaheen}
\date{\parbox{\linewidth}{\centering
  \endgraf\bigskip
  University of Oulu
  \endgraf\bigskip\today}}
}

\setlength{\parindent}{0pt}

\begin{document}
\maketitle
\newpage

\section{Abstract}
Household waste has been around throughout the whole humanity lifetime. In current document we
elaborate an idea of (semi-)automated waste management system that aims to simplify everyday human
interaction with wastes. Our idea provides a comprehensive solution to waste management problem.
Multiple tiers of the system are observed to focus both on implementations that are possible today
and also on some that target future technologies.

\section{Concept}

The problem in focus is the waste and the necessity to manage waste utilization and disposal. As a
number of people increases, more waste is produced and requirements for waste management become more
strict and more challenging to satisfy.

Tranditionally, waste management was a manual procedure with different (usually non-optimal) parts.
For example, a person is required to periodically check whether waste bins are full and most likely
remove the content of the bins whether they are empty, half-full or completely full. Not only more
time is spent in this example, there are also additional costs e.g. due to fuel usage (likely a
vehicle is utilized to collect the waste).

Our solution aims to reduce the time and costs spent on waste collection with further improvement in
terms of automation. We believe that the system is beneficial mostly to the city residents rather
than to visitors.

\section{System tiers}

In this section we introduce multiple tiers of the same system so that the reader has better
understanding of possibilities of each tier. Later chapters are explained in terms of the introduced
tiers.

\subsection{Tier 1: Smart waste bins}

This tier is the basic tier of the system. The focus of the system in context of this tier is to
provide "smart" waste bins. The smart waste bin has the following components:

- There is an attached IoT device (e.g. Arduino or other microcontroller) and the device has a
wireless network connection (WiFi, Internet, etc.)

- The waste bin is able to indentify whether it is full or close to being full. This is achieved by
a sensor (e.g. infrared) that monitors the number of waste inside the bin and reports the
measurements to the IoT device for processing

The system consists of a multitude of smart waste bins and a server-side application that aggregates
information from the waste bins (whether they are full or not) and reports the waste bin statuses to
the user of the system (here, a human).

\subsection{Tier 2: Locally automated waste management}

This tier is based on the previous Tier 1. More automation takes place in this system by utilizing
the robots to collect waste from individual smart waste bins. Thus, instead of using a server-side
application waste bins the information is sent to a local robot which is responsible for waste
collection and waste delivery to the centralized waste storage (an ultimate waste bin in a sense).

The centralized waste storage is a bigger waste collection place that aggregates waste from multiple
households. There can exist one such storage per residential area or even one for the whole
district. The main idea is that such storage is easily accessible by vehicles and requires less time
to collect the waste from. At this level, the management of the centralized waste storage is manual
to a certain degree.

\subsection{Tier 3: Globally automated waste management}

As before, this tier is based on the Tier 2 system. The major difference is a leap from
semi-automated solution to fully automated. The centralized waste storage briefly introduced above
is now fully automated: we rely on autonomous vehicles and extra robotics to take care of waste
aggregation and waste disposal by having similar intelligence, as in the case of smart waste bins,
for the centralized waste storage. The storage monitors the volume of the waste and notifies
autonomous vehicles when the waste collection is required. Such vehicles are responsible for waste
transportation from storages to the waste processing plants.

\section{Use case}

Describe the audience or users of the application and area or place where the new solution takes
place. Describe devices involved (e.g. personal devices, public screens and other user interfaces,
sensors, computing units, etc). Describe users who need the solution and how they benefit using the
solutions. Use pictures and drawings to show operation of your solution and/or users using it.

\subsection{Tier 1: Smart waste bins}

% An individual waste bin is self-aware and continuously monitors whether it is full or not. Once it
% is full, the microcontroller sends a corresponding message over the network. On the other side of
% the network, the message is read and addressed: if the waste bin is full, there is a need to
% collect the waste. The solution scales to many waste bins in a similar fashion. Additional logic
% may be present if there are waste bins of different types (e.g. biowaste, energy waste) or if
% there are multiple waste bins (e.g. one is full, the other is not).

\subsection{Tier 2: Locally automated waste management}

\subsection{Tier 3: Globally automated waste management}


\section{Methodology}
Propose a technical solution for implementation of your design. How do the participating devices
communicate? What are sensors and actuators your design uses? What kind of data is collected, and
how it needs to be processed? Where does the computation happen (in local devices, cloud, somewhere
else)? Use graphs and images.

\section{Technical details}
TODO: add solution diagram?

\section{Technical challenges}
TODO: challenges with bin fulfillment check (imperfections in IR, etc)

\section{Outcomes and conclusion}


\end{document}
